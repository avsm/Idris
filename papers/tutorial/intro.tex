\section{Introduction}

\Idris{} is an experimental functional programming language with full
dependent types. This document is intended as a brief introduction to
the language, and is aimed at readers already familiar with a
functional language such as Haskell or OCaml.

\subsection{What are dependent types?}

\subsection{Related languages}

Agda, Epigram and Coq. The distinctive feature of \Idris{} is that it
has both dependent pattern matching (like Agda and Epigram) and an
extensible tactic-based theorem prover (like Coq) in an effort to reap
the benefits of both. It serves as an environment for exploring the
design space of dependently typed functional programming languages,
without being limited by design decisions of other more mature
systems.

\subsection{Downloading and Installation}

\Idris{} is available from \url{http://www.cs.st-and.ac.uk/~eb/Idris}.
You can download a package containing the latest version from:

\UL{
\item \url{http://www-fp.cs.st-and.ac.uk/~eb/idris/idris-current.tgz}.
}

To install, you will require GHC 6.10 or greater, and the Boehm
garbage collector (available from
\url{http://www.hpl.hp.com/personal/Hans_Boehm/gc/}, or as a package
from Macports or your favourite Linux distribution).

Download the package, untar it with ``\texttt{tar zxvf
  idris-current.tgz}'', \texttt{cd} to the new directory (which will
have a name of the form \texttt{idris-date}, where \texttt{date} gives
the date the package was created, then type ``\texttt{make}''. This
will install the \texttt{idris} executable in your home
directory. 

Ensure that \texttt{\~{ }/bin} is in your \texttt{\$PATH}. If
everything has worked, you should be able to execute \texttt{idris},
and see something like:

\begin{verbatim}
$ idris
Idris version 0.1.2
-------------------
Usage:
	idris <source file> [command]
\end{verbatim}


