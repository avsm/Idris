\section{Examples}

Datatypes are defined using a similar syntax to Haskell. For simple or
polymorphic types, we use the usual syntax. However, since type
constructors and value constructors inhabit the same namespace, we
must choose distinct names. Additionally, constructors need not begin
with a capital letter. For example:

\DM{
\AR{
\Data\hg\Nat\:=\:\Z\:\mid\suc\:\Nat\hg{\mbox{\texttt{-- natural numbers}}}\\
%\Data\hg\TC{Tree}\:\va\:=\:\DC{Empty}\:\mid\:\DC{Node}\:(\TC{Tree}\:\va)\:\va\:(\TC{Tree}\:\va)\hg{\mbox{\texttt{-- binary trees}}}
\Data\hg\TC{List}\:\va\:=\:\DC{Nil}\:\mid\:\DC{Cons}\:\va\:(\List\:\va)
\hg{\mbox{\texttt{-- linked lists}}}}
}

Dependent types are declared using a syntax similar to Haskell's
GADTs, for example in the following definition of vectors (lists with
size) we give the type of $\Vect$ explicitly, where $\Type$ is the
type of types, and the type of each constructor separately:

\DM{\AR{
\Data\hg\:\Vect\:\Hab\:\AR{\Type\to\Nat\to\Type\hg\Where}\\ 
\begin{array}{rl}
 & \Vnil\:\Hab\:\AR{\Vect\:\VV{A}\:\Z}\\ 
 \mid & \Vcons\:\Hab\:\AR{\VV{A}\to\Vect\:\VV{A}\:\VV{k}\to\Vect\:\VV{A}\:(\suc\:\VV{k})}
\end{array}
}}

Functions are defined by pattern matching. Top level types must always
be given (expect for constants, i.e. functions with zero arguments)
since full type inference for any dependently typed language is
undecidable. Type annotations on $\LET$ or lambda bindings are
generally not necessary, but may be given.

For example, the following function is a dependent version of
$\FN{zipWith}$, which applies a function to corresponding elements in a
pair of vectors:

\DM{\AR{
\FN{zipWith}\:\Hab\:\AR{(\VV{f} \Hab \VV{A}\to\VV{B}\to\VV{C})\to\Vect\:\VV{A}\:\VV{n}\to\Vect\:\VV{B}\:\VV{n}\to\Vect\:\VV{C}\:\VV{n}}\\ 
\PA{\A\A\A}{ & \FN{zipWith} & \VV{f} & \Vnil & \Vnil & \Ret{\Vnil}\\ 
 & \FN{zipWith} & \VV{f} & (\Vcons\:\VV{x}\:\VV{xs}) & (\Vcons\:\VV{y}\:\VV{ys}) & \Ret{\Vcons\:(\VV{f}\:\VV{x}\:\VV{y})\:(\FN{zipWith}\:\VV{f}\:\VV{xs}\:\VV{ys})}\\ 
}}}

Note that we do not handle any cases where the vectors are difference
sizes; the type ensures that such cases are not possible, since the
input and output vectors are statically known to be the same size.

In $\FN{zipWith}$, the variables $\vA$, $\vB$, $\vC$ and $\vn$ are
\remph{implicit arguments} to $\FN{zipWith}$. The typechecker
establishes, by generating and solving a unification problem, that
$\vA$, $\vB$ and $\vC$ must have type $\Type$ and $\vn$ must have type
$\Nat$. It is sometimes necessary to write such arguments explicitly,
to help solve the unification problem, where these arguments
themselves have dependent types. The type of $\FN{zipWith}$
could therefore also be written as:

\DM{
\FN{zipWith}\:\Hab\:\AR{
\{\vA,\vB,\vC\Hab\Type\}\to\{\vn\Hab\Nat\}\to\\
(\VV{f} \Hab \VV{A}\to\VV{B}\to\VV{C})\to\Vect\:\VV{A}\:\VV{n}\to\Vect\:\VV{B}\:\VV{n}\to\Vect\:\VV{C}\:\VV{n}}\\ 
}

\noindent
The use of braces $\{\}$ rather than brackets $()$ indicates that
these arguments need not be given in calls to $\FN{zipWith}$.

