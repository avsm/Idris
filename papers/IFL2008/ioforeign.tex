\section{I/O and Foreign Functions}

\label{ioforeign}

\newcommand{\IO}{\TC{IO}}
\newcommand{\IODo}{\DC{IODo}}
\newcommand{\IOReturn}{\DC{Return}}

We require input/output operations to be executable from within our
language. To achieve this, we use Hancock and Setzer's I/O
model~\cite{hancock-interactive}. This involves implementing a
Haskell-style I/O monad~\cite{HaskellIO} by defining \remph{commands},
representing externally executed I/O operations, and \remph{responses}
which give the type of the value returned by a given command. Our
$\IO$ monad is then implemented in terms of these of commands and
responses:

\DM{\AR{
\Data\hg\:\TC{IO}\:\Hab\:\Type\to\Type\hg\Where\\ 
\begin{array}{rl}
 & \DC{IOReturn}\:\Hab\:\VV{A}\to\TC{IO}\:\VV{A}\\ 
 \mid & \DC{IODo}\:\Hab\:\AR{(c \Hab
 \TC{Command})\to\\
(\FN{Response}\:\VV{c}\to\TC{IO}\:\VV{A})\to \TC{IO}\:\VV{A}}
\end{array}
}}

\noindent
Here, $\IODo$ takes a command, $\vc$ and an I/O operation that transforms the response to that command into an $\IO$ value, 
and returns the result of applying the I/O operation to the command; and $\IOReturn$ simply returns an action packaged as an $\IO$ value.
To show how this works in practice, we can define simple commands and
responses for reading and writing to standard input and output:

\DM{\AR{
\Data\hg\:\TC{Command}\:\Hab\:\Type\hg\Where\\ 
\begin{array}{rl}
 & \DC{PutStr}\:\Hab\:\TC{String}\to\TC{Command}\\ 
 \mid & \DC{GetStr}\:\Hab\:\TC{Command}\\ 
 \mid & \ldots 
\end{array}
}}

\noindent
The responses to these commands are a $\TC{String}$, in the case that
we have read a string, or the unit type if we have written a string.

\DM{\AR{
\FN{Response}\:\Hab\:\AR{\TC{Command}\to\Type}\\ 
\PA{\A}{ & \FN{Response} & (\DC{PutStr}\:\VV{s}) & \Ret{()}\\ 
 & \FN{Response} & \DC{GetStr} & \Ret{\TC{String}}\\ 
 & \ldots
}
}}

\newcommand{\ioexec}{\MO{execute}}
\newcommand{\iorun}{\MO{run}}
\newcommand{\iodo}[1]{\RW{do}\;\AR{#1}}
\newcommand{\ioreturn}{\RW{return}}
\newcommand{\iosep}{\\}

\noindent
The usual $\FN{bind}$ and $\FN{return}$ operations are simple to
implement for this monad. We can now write the higher-level reading
and writing operations:

\DM{
\AR{
\FN{getStr}\Hab\IO\:\TC{String}\\
\FN{getStr}\:=\:\IODo\:\DC{GetStr}\:(\lam{\vs}{\TC{String}}\SC\IOReturn\:\vs)
\\
\FN{putStr}\Hab\TC{String}\to\IO\:\Unit\\
\FN{putStr}\:\vs\:=\:\IODo\:(\DC{PutStr}\:\vs)\:(\lam{\vx}{\TC{Unit}}\SC\IOReturn\:\vx)
}
}

\noindent
Execution of an I/O program consists of evaluating it as normal and
passing the result to an $\ioexec$ meta-operation. This is defined
externally to the type theory and simply executes the I/O action at
the head of the term, then evaluates and executes the
continuation. Execution of the above simple string I/O
language is defined by the following pseudo-Haskell code (for
convenience, we will use Haskell-style $\RW{do}$ notation, with the obvious translation into our $\IODo$ and $\IOReturn$ operations):

\DM{
\AR{
\PA{\A}{
& \ioexec & (\IODo\:\vc\:\vr) & \MoRet{\iorun\:\vc\:\vr} \\
& \ioexec & (\IOReturn\:\vv) & \MoRet{\ioreturn\:\vv} 
}
\smallskip\\
%}
%\DM{
\PA{\A\A}{
& \iorun & \DC{PutStr} & \vk & \MoRet{\iodo{
   \mathtt{str} \leftarrow \mathtt{getLine}\\
   \ioexec\:(\vk\:\mathtt{str})
   }} \\
& \iorun & (\DC{GetStr}\:\vs) & \vk & \MoRet{\iodo{
   \mathtt{putStr}\:\vs\\
   \ioexec\:(\vk\:\UnitI)
   }} \\
& \ldots
}
}
}

In our compiler, which translates \Idris{} to C, we implement this
operation with a compiler pass which partially evaluates the
$\FN{bind}$ operation and translates the commands into their C
equivalents.

%% For the program we will implement in this paper, we require the
%% following operations, implemented similarly to $\FN{getStr}$ and
%% $\FN{putStr}$ above:

%% \DM{
%% \AR{
%% \FN{fork}\:\Hab\:\AR{\IO\:()\to\IO\:()}\\
%% \FN{newLock}\:\Hab\:\AR{\TC{Int}\to\IO\:\TC{Lock}}\\ 
%% \FN{lock},\FN{unlock}\:\Hab\:\AR{\TC{Lock}\to\IO\:()}\\ 
%% \FN{newIORef}\:\Hab\:\AR{\VV{A}\to\IO\:(\TC{IORef}\:\VV{A})}\\ 
%% \FN{readIORef}\:\Hab\:\AR{\TC{IORef}\:\VV{A}\to\IO\:\VV{A}}\\ 
%% \FN{writeIORef}\:\Hab\:\AR{\TC{IORef}\:\VV{A}\to\VV{A}\to\IO\:()}\\ 
%% }
%% }

%% $\TC{Lock}$ is an externally implemented semaphore, created by
%% $\FN{newLock}$, with $\FN{lock}$ and $\FN{unlock}$ requesting and
%% releasing the semaphore respectively. The other functions behave in
%% the same way as their Haskell equivalents.

\subsection{Foreign Function Calls}

The $\TC{IO}$ monad as defined above requires us to decide which
external operations are allowed, and to implement the appropriate
$\MO{run}$ operation. While this does allow us to write realistic
programs which communicate with the outside world, it lacks
flexibility --- there is no way to access arbitrary C libraries for,
e.g., concurrency, graphics, or networking without modifying the
implementation of $\MO{run}$. To achieve this, we introduce a
representation of external functions, and a means to convert them into
usable \Idris{} functions. We being by defining the C types which we
can use, and a conversion into \Idris{} types:

\DM{\AR{
\Data\hg\:\TC{FType}\:\Hab\:\AR{\Type\hg\Where}\\ \hg 
\DC{FUnit}\:\Hab\:\AR{\TC{FType}}\: 
 \mid \DC{FInt}\:\Hab\:\AR{\TC{FType}}\: 
 \mid \DC{FStr}\:\Hab\:\AR{\TC{FType}}\: 
 \mid \DC{FPtr}\:\Hab\:\AR{\TC{FType}}
}}

\DM{\AR{
\FN{i\_ftype}\:\Hab\:\AR{\TC{FType}\to\Type}\\ 
\PA{\A}{ & \FN{i\_ftype} & \DC{FInt} & \Ret{\TC{Int}}\\ 
 & \FN{i\_ftype} & \DC{FStr} & \Ret{\TC{String}}\\ 
 & \FN{i\_ftype} & \DC{FUnit} & \Ret{()}\\ 
}}}

Foreign pointers are treated as abstract entities; there is no means
to modify them, they can simply be passed to and from C. A foreign
function can then be described using an ordinary data structure. It
has a name, a list of argument types and a return type:

\DM{\AR{
\Data\hg\:\TC{ForeignFun}\:\Hab\:\AR{\Type\hg\Where}\\ 
\begin{array}{rl}
 & \DC{FFun}\:\Hab\:\AR{\TC{String}\to\TC{List}\:\TC{FType}\to\TC{FType}\to\TC{ForeignFun}}
\end{array}
}}

When we call the foreign function, it will need an argument list. We
can represent this as a type depending on the list of argument
types. This allows us to typecheck foreign function calls in the core
language itself:

\DM{\AR{
\Data\hg\:\TC{FArgList}\:\Hab\:\AR{\TC{List}\:\TC{FType}\to\Type\hg\Where}\\ 
\begin{array}{rl}
 & \DC{fNil}\:\Hab\:\AR{\TC{FArgList}\:\DC{Nil}}\\ 
 \mid & \DC{fCons}\:\Hab\:\AR{
\{\vx\Hab\TC{FType}\}\to\{\vxs\Hab\TC{List}\:\TC{FType}\}\to\\
(\VV{fx} \Hab \FN{i\_ftype}\:\VV{x})\to(\VV{fxs} \Hab \TC{FArgList}\:\VV{xs})\to\TC{FArgList}\:(\DC{Cons}\:\VV{x}\:\VV{xs})}
\end{array}
}}

Since we now have descriptions of foreign functions, and a means to
give them a typechecked list of arguments, we can add foreign function
calls to our $\TC{Command}$ structure.

\DM{\AR{
\Data\hg\:\TC{Command}\:\Hab\:\AR{\Type\hg\Where}\\ 
\begin{array}{rl}
& \ldots \\
 \mid & \DC{Foreign}\:\Hab\:\AR{(\VV{f} \Hab \TC{ForeignFun})\to(\VV{args} \Hab \TC{FArgList}\:(\FN{f\_args}\:\VV{f}))\to\TC{Command}}
\end{array}
}}

The only extension we need make to the system is in the $\MO{execute}$
and $\MO{run}$ operations. We will not give the full details here, but
it needs only a means of translation between the run-time
representation of \Idris{} values and their corresponding C
representation. The C function to call is given in the
$\TC{ForeignFun}$ structure, so the execution is straightforward.

\subsection{Constructing foreign functions}

Although we can represent foreign functions using the above
structures, it is inconvenient to require a programmer to use the
$\TC{Command}$ structure directly. We would like to be able to convert
the above representation into a more convenient form. We will do this
by converting the foreign type into an \Idris{} type, and building a
corresponding function. Reflection of this kind is implementable
directly in a dependently typed language. For example, given a foreign
function described as\ldots

\DM{\FN{printFn}\:=\:\DC{FFun}\:"putStr"\:(\DC{Cons}\:\DC{FStr}\:\DC{Nil})\:\DC{FUnit}}

\noindent
\ldots we would like a corresponding function to expose to the user.

\DM{\FN{print}\Hab\TC{String}\to\TC{IO}\:()}

To construct the type of this function, we write a helper function
$\FN{mkType}$ which takes a list of argument types and a return type,
and from these constructs a function type in the $\TC{IO}$ monad:

\DM{\AR{
\FN{mkFType}\:\Hab\:\AR{\TC{List}\:\TC{FType}\to\TC{FType}\to\Type}\\ 
\PA{\A\A}{ & \FN{mkFType'} & \DC{Nil} & \VV{ret} & \Ret{\TC{IO}\:(\FN{i\_ftype}\:\VV{ret})}\\ 
 & \FN{mkFType'} & (\DC{Cons}\:\VV{t}\:\VV{ts}) & \VV{ret} & \Ret{\FN{i\_ftype}\:\VV{t}\to\FN{mkFType'}\:\VV{ts}\:\VV{ret}}\\ 
}}}

A function of this type will need to construct the appropriate
$\DC{Foreign}$ command, which involves building an appropriate
$\TC{FArgList}$ from its input. We compute a function which builds
this argument list in an accumulator, iterating over the list of
argument types:

\DM{\AR{
\FN{mkFDef}\:\Hab\:\AR{\TC{String}\to(\VV{ts} \Hab
  \TC{List}\:\TC{FType})\to(\VV{xs} \Hab \TC{List}\:\TC{FType})\to\\
\TC{FArgList}\:\VV{xs}\to(\VV{ret} \Hab \TC{FType})\to\FN{mkFType}\:\VV{ts}\:\VV{ret}}\\ 
\PA{\A\A\A\A\A}{ & \FN{mkFDef} & \VV{nm} & \DC{Nil} & \VV{accA} &
  \VV{fs} & \VV{ret} & \\
& & & \MRet{5}{\DC{IODo}\:(\DC{Foreign}\:(\DC{FFun}\:\VV{nm}\:\VV{accA}\:\VV{ret})\:\VV{fs})\:(\lambda\VV{a}.\DC{IOReturn}\:\VV{a})}\\ 
 & \FN{mkFDef} & \VV{nm} & (\DC{Cons}\:\VV{t}\:\VV{ts}) & \VV{accA} &
  \VV{fs} & \VV{ret} & \\
& & & \MRet{5}{\lambda\VV{x}.\FN{mkFDef}\:\VV{nm}\:\VV{ts}\:(\DC{Cons}\:\VV{t}\:\VV{accA})\:(\DC{fCons}\:\VV{x}\:\VV{fs})\:\VV{ret}}\\ 
}}}

We can now make the appropriate \Idris{} function from a
$\TC{ForeignFun}$ definition as follows:

\DM{\AR{
\FN{mkForeign}\:\Hab\:\AR{(\VV{f} \Hab \TC{ForeignFun})\to\FN{mkFType}\:\VV{f}}\\ 
\PA{\A}{ & \FN{mkForeign} & (\DC{FFun}\:\VV{fn}\:\VV{args}\:\VV{ret}) & \Ret{\FN{mkFDef}\:\VV{fn}\:\VV{args}\:\DC{Nil}\:\DC{fNil}\:\VV{ret}}\\ 
}}}

Using these, we can now declare the existence of a foreign function
called \texttt{putStr} --- we assume the existence of this in a C
library, which simply outputs a string and returns \texttt{void}:

\DM{\FN{printFn}\:=\:\DC{FFun}\:"putStr"\:(\DC{Cons}\:\DC{FStr}\:\DC{Nil})\:\DC{FUnit}}

\noindent
We now use $\FN{mkForeign}$ to turn this into an \Idris{} function:

\DM{\AR{
\FN{print}\:\Hab\:\AR{\TC{String}\to\TC{IO}\:()}\\ 
\PA{}{ & \FN{print} & \Ret{\FN{mkForeign}\:\FN{printFn}}\\ 
}}}

A programmer writing a binding to a C library then need only use
$\DC{FFun}$ and $\FN{mkForeign}$ to make a type safe \Idris{} binding
to a C function.
